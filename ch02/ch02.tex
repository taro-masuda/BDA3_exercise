\documentclass[pdflatex,ja=standard]{bxjsarticle}

% Language setting
% Replace `english' with e.g. `spanish' to change the document language
\usepackage[japanese]{babel}
\usepackage{graphicx} % Required for inserting images
\usepackage{amsmath}
\usepackage{amssymb}
\usepackage{xeCJK}
\usepackage{listings}
\usepackage{autobreak}
\begin{document}

2.3

(a)

$y \sim \rm{Bin}(1000, 1/6) $であるため $y \sim \mathcal{N} (1000/6, 1000/6\cdot(1-1/6))$
と近似できる。

\begin{lstlisting}
import numpy as np
import matplotlib.pyplot as plt

def norm_dist(x: np.ndarray, mu: float, sigma: float) -> np.ndarray:
    return 1 / (sigma * np.sqrt(2 * np.pi)) * np.exp(- 1./2 * ((x - mu) / sigma) **2)

#%% 
x = np.linspace(0, 300, 1000)
y = norm_dist(x, 1000/6, np.sqrt(1000/6 * 5/6))

plt.plot(x, y)
plt.show()

\end{lstlisting}

\begin{figure}
    \centering
    \includegraphics[width=0.5\linewidth]{1-3.png}
    \caption{2.3の描画結果}
    \label{fig:placeholder}
\end{figure}

(b)

\begin{lstlisting}
# %%
from scipy.stats import norm
l = [norm.ppf(x)*np.sqrt(1000/6 * 5/6) + 1000/6 for x in [0.05, 0.25, 0.5, 0.75, 0.95]]
print([float(x) for x in l])
\end{lstlisting}

答えはそれぞれ [147.3, 158.7, 166.7, 174.6, 186.0]。

2.15

\begin{gather}
\mathbb{E} [Z^m(1-Z)^n] = \frac{\Gamma(\alpha+\beta)}{\Gamma(\alpha)\Gamma(\beta)} \int_0^1 z^m (1-z)^n z^{\alpha-1}(1-z)^{\beta-1} dz \\
= \frac{\Gamma(\alpha+\beta)}{\Gamma(\alpha)\Gamma(\beta)} \int_0^1 z^{m+\alpha-1} (1-z)^{n+\beta-1} dz \\
= \frac{\Gamma(\alpha+\beta)\Gamma(\alpha+m)\Gamma(\beta+n)}{\Gamma(\alpha) \Gamma(\beta)\Gamma(\alpha+\beta+m+n)} \\
=\frac{(\alpha+m-1)\cdots \alpha \Gamma(\alpha) (\beta+n-1) \cdots \beta \Gamma(\beta) \Gamma(\alpha+\beta)}{\Gamma(\alpha)\Gamma(\beta)(\alpha+\beta+m+n-1) \cdots (\alpha+\beta) \Gamma(\alpha+\beta)} \\
=\frac{(\alpha+m-1) \cdots \alpha (\beta+n-1) \cdots\beta}{(\alpha+\beta+m+n-1) \cdots (\alpha+\beta)}
\end{gather}
よって、平均は
\begin{equation}
\mathbb{E}[Z^1(1-Z)^0]=\frac{\alpha}{\alpha+\beta}
\end{equation}
分散は
\begin{gather}
\mathbb{E}[Z^2(1-Z)^0] - (\mathbb{E}[Z])^2 = \frac{(\alpha+1)\alpha}{(\alpha+\beta+1)(\alpha+\beta)} - \frac{\alpha^2}{(\alpha+\beta)^2} \\
= \frac{(\alpha^2+\alpha)(\alpha+\beta) - \alpha^3 -\alpha^2\beta-\alpha^2}{(\alpha+\beta+1)(\alpha+\beta)^2} \\
= \frac{\alpha\beta}{(\alpha+\beta+1)(\alpha+\beta)^2}
\end{gather}

\end{document}
